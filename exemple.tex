\documentclass{beamer}

\mode<presentation> {

% The Beamer class comes with a number of default slide themes
% which change the colors and layouts of slides. Below this is a list
% of all the themes, uncomment each in turn to see what they look like.

%\usetheme{default}
%\usetheme{AnnArbor}
%\usetheme{Antibes}
%\usetheme{Bergen}
%\usetheme{Berkeley}
%\usetheme{Berlin}
%\usetheme{Boadilla}
%\usetheme{CambridgeUS}
%\usetheme{Copenhagen}
%\usetheme{Darmstadt}
%\usetheme{Dresden}
%\usetheme{Frankfurt}
%\usetheme{Goettingen}
%\usetheme{Hannover}
%\usetheme{Ilmenau}
%\usetheme{JuanLesPins}
%\usetheme{Luebeck}
\usepackage{verbatim}
\usetheme{Madrid}
%\usetheme{Malmoe}
%\usetheme{Marburg}
%\usetheme{Montpellier}
%\usetheme{PaloAlto}
%\usetheme{Pittsburgh}
%\usetheme{Rochester}
%\usetheme{Singapore}
%\usetheme{Szeged}
%\usetheme{Warsaw}

% As well as themes, the Beamer class has a number of color themes
% for any slide theme. Uncomment each of these in turn to see how it
% changes the colors of your current slide theme.

%\usecolortheme{albatross}
%\usecolortheme{beaver}
%\usecolortheme{beetle}
%\usecolortheme{crane}
%\usecolortheme{dolphin}
%\usecolortheme{dove}
%\usecolortheme{fly}
%\usecolortheme{lily}
%\usecolortheme{orchid}
%\usecolortheme{rose}
%\usecolortheme{seagull}
%\usecolortheme{seahorse}
%\usecolortheme{whale}
%\usecolortheme{wolverine}

%\setbeamertemplate{footline} % To remove the footer line in all slides uncomment this line
%\setbeamertemplate{footline}[page number] % To replace the footer line in all slides with a simple slide count uncomment this line

%\setbeamertemplate{navigation symbols}{} % To remove the navigation symbols from the bottom of all slides uncomment this line
}
\usepackage{amsmath}
\usepackage{indentfirst}
\usepackage{float}
\usepackage{longtable}
\usepackage{CJKutf8}
\usepackage{biblatex}
\usepackage{graphicx} % Allows including images
\usepackage{booktabs} % Allows the use of \toprule, \midrule and \bottomrule in tables

%----------------------------------------------------------------------------------------
%	TITLE PAGE
%----------------------------------------------------------------------------------------


\setbeamertemplate{caption}[numbered]



\begin{document}


\title[2021年春季学期科研规划]{2021年春季学期科研规划} % The short title appears at the bottom of every slide, the full title is only on the title page

\author{刘勇} % Your name
\institute[北京大学重离子物理研究所] % Your institution as it will appear on the bottom of every slide, may be shorthand to save space
{
北京大学重离子物理研究所 \\ % Your institution for the title page
\medskip
导师: 薛建明 % Your email address
}
\date{2021年3月19日} % Date, can be changed to a custom date

\begin{CJK*}{UTF8}{gbsn}

\begin{frame}
\titlepage % Print the title page as the first slide
\end{frame}



%------------------------------------------------


\begin{frame}
\frametitle{总体规划}
\begin{enumerate}
    \item 文献调研
    \item 软件学习
    \item 实践
\end{enumerate}
\end{frame}


%----------想要--------------------------------------

%------------------------------------------------

\begin{frame}
\frametitle{文献调研}
    
\end{frame}

%-------------------------------------------

\begin{frame}
\frametitle{软件学习}

\end{frame}



\begin{frame}
\frametitle{计算实践}

\end{frame}


%---------------------------------------------------------------------------------------
\end{CJK*}
\end{document}



%%%%%%%%%%%%%%%%%%%%%%%%%%%%%%%%%%%%%%%%%%%%%%%%%%%%%%%%%%%%%%%%%%%%%%%%%%%%%%%%%%%%%%%%%%%%%%%%%%%%%%%%%%%%%%%%%%%%%%%%%%%%%%%%%%%%%%%%%%%%%%%%%%%%%%%%%%%%%%%%%%%%%%%%%%%%%%%%%%%%%%%%%%%%%%%%%%%%%%%%%%%%%%%%%%%%%%%%%%%%%%%%%%%%%%%%%%%%%%%%%%%%%%%%%%%%%%%%%%%%%%%%%%%%%%%%%%%%%%%%%%%%%%%%%%%%%%%%%%%%%%%%%%%%%%%%%%%%%%%%%%%%%%%%%%%%%%%%%%%%%%%%%%%%%%%%%%

\begin{frame}
\frametitle{Bibliography}

\footnotesize{
\begin{thebibliography}{99} % Beamer does not support BibTeX so references must be inserted manually as below
\bibitem[Hao Wang,Xun Guo,Linfeng Zhang,Han Wang,and Jianming Xue]{p1}Hao Wang,Xun Guo,Linfeng Zhang,Han Wang,and Jianming Xue
\newblock Deep learning inter-atomic potential model for accurate irradiation damage simulations
\newblock \emph{Applied Physics Letters} 114,244101(2019).
\end{thebibliography}
}

\footnotesize{
\begin{thebibliography}{99} % Beamer does not support BibTeX so references must be inserted manually as below
\bibitem[Jiequn Han,Linfeng Zhang,Roberto Car and Weinan E]{p1}Jiequn Han,Linfeng Zhang,Roberto Car and Weinan E
\newblock Active learning of uniformly accurate interatomic potential for material simulation
\newblock \emph{Physical review materials} 3,023804(2019).
\end{thebibliography}
}

\footnotesize{
\begin{thebibliography}{99} % Beamer does not support BibTeX so references must be inserted manually as below
\bibitem[Jiequn Han,Linfeng Zhang,Roberto Car and Weinan E]{p1} Jiequn Han,Linfeng Zhang,Roberto Car and Weinan E
\newblock Deep Potential:a genneral representation of a many-body potential energy surface
\newblock \emph{Commun. Comput. Phys} 18(3), 623 -- 639.
\end{thebibliography}
}


\footnotesize{
\begin{thebibliography}{99} % Beamer does not support BibTeX so references must be inserted manually as below
\bibitem[Linfeng Zhang,Jiequn Han,Han Wang,Wissam A.Saidi,Roberto Car,and Weinan E]{p1}Linfeng Zhang,Jiequn Han,Han Wang,Wissam A.Saidi,Roberto Car,and Weinan E
\newblock End-to-end Symmetry Preserving Inter-atomic Potential Energy Model for finite and extended system
\newblock \emph{arXiv}:1805.09003 [physics.comp-ph]
\end{thebibliography}
}


\end{frame}


\begin{frame}
\Large{\centerline{The End}}
\Huge{\centerline{Thank You}}
\end{frame}